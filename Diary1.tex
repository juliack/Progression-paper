\documentclass{letter}
\usepackage[utf8]{inputenc}
\usepackage{amsmath}
\usepackage{graphicx} %for images
\usepackage[utf8]{inputenc} %for ToC
\usepackage{lscape}%for landscape layout
\usepackage{rotating}%to rotate figs
\usepackage{geometry}%to change margins
\usepackage{hyperref} %for links
\hypersetup{
linktoc=all,  
    colorlinks,
    citecolor=black,
    filecolor=black,
    linkcolor=black,
    urlcolor=black
}

%for bibliography"
%with biblatex
%\usepackage[style=authoryear]{biblatex}
%\addbibresource{references.bib}
% or with natbib
\usepackage{natbib} 
\usepackage{har2nat}
%\bibliography{references.bib}

\begin{document}
\pagenumbering{roman}

Last week, I submitted the most substantial piece of writing of my PhD so far. Besides the normal challenges that come with absorbing a breadth of literature and demonstrating enough understanding of it to be able to plan something resembling an achievable research project, one unexpected but major achievement that I scored with this opportunity was to adopt a new skill - editing in LateX. 

When solving any problem, I am fond of saying that the shortest road is the road you know - and I have learned this the hard way when going out traipsing with my bike or returning from a run in the countryside, time and time again... and again... and again. 

However, the temptation is always there. And truth be told, if you are prepared for a longer ride than expected (with a charged headlamp, phone and water), sometimes new discoveries are well worth it, and may become a regular fixture for your future endeavors to the point that you wonder why it took you so long to explore this inviting side-road.  

On the other hand, in adopting new software solutions, the temptation of uncharted territory is, in my experience, more often replaced by skepticism and resistance. Especially because almost any software comes with its early adoptees who are apt to evangelize and instill in you a defensiveness of your old habits that resembles sports team partisanship. The time investment necessary to master a new skill is also a barrier when you already posses a 'good enough' solution. But unless you overcome that hump, there is no way to know which advancements would really change your workflow for the better and what new possibilities they may reveal. In this case, it seems advisable to every now and then challenge whether the road you know is really the best one and whether it just may be the right time to take the plunge and put in the time to pick up that new skill you've heard everyone talk about.

\end{document}
 